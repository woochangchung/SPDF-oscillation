\documentclass[aps,prl,twocolumn,superscriptaddress]{revtex4-1}
%\documentclass[aps,prl,preprint,superscriptaddress]{revtex4-1}
%\documentclass[aps,prl,reprint,groupedaddress]{revtex4-1}

%%%%%%%%%%%%%%%%%%%%%%%%%%%%%%%%%%%%%%%%%%%%%%%%%%%%%%%%%%%%%%%%%%%%%%%%%%%%%%%%%%%%%%%%%%%%%%%%%%
% Some things to consider while writing:
%
%  - We should agree on one notation to distinguish between the Rb hyperfine basis and the S = 1 basis. I've regularly found myself using |1, 1> meaning |S = 1, m_S = 1>, but this could also be interpreted as |F = 1, m_F = 1>. My suggestion is that we reserve |A> and |B> for the two hyperfine spin components since they change depending on whether we use u > 0 or u < 0, and that we maybe use the double arrows to refer to the spin basis (|⇑>).
%  - For general notation matters see the Physical Review Styleguide: https://cdn.journals.aps.org/files/styleguide-pr.pdf
%



%  - In terms of content:
%       * During the time evolution we evolve into an entangled state. How do we mention this?
%       * To what extent do we talk about thermalization in the spin sector?
%%%%%%%%%%%%%%%%%%%%%%%%%%%%%%%%%%%%%%%%%%%%%%%%%%%%%%%%%%%%%%%%%%%%%%%%%%%%%%%%%%%%%%%%%%%%%%%%%%

\usepackage[usenames,dvipsnames,svgnames,table]{xcolor}
\usepackage{graphicx}
\usepackage{amsmath}
\usepackage{blindtext}
\usepackage{printlen}
\usepackage{hyperref}
\usepackage{braket}
\usepackage{todonotes}
\uselengthunit{in}

\newcounter{comment}
\newcommand{\comment}[2][]{\todo[color=red!100!green!33, #1]{#2}}
\definecolor{myellow}{rgb}{1., 1., 0.6}
\newcommand{\note}[2][]{\todo[color=myellow, #1]{#2}}



\begin{document}

\title{Quantum Simulation of Spin-pairing Dynamics with Two-component Bosons in Doubly-occupied Lattice Sites}
% ?? Alternative, throwing in some buzzwords: Quantum Simulation of Superexchange-driven Nonequilibrium Dynamics in the $S = 1$ Heisenberg model
            
\author{BEC4}
\affiliation{Research Laboratory of Electronics, MIT-Harvard Center for Ultracold Atoms, Department of Physics, Massachusetts Institute of Technology, Cambridge, Massachusetts 02139, USA}

\date{\today}

\begin{abstract}
Low-energy physics of doubly-occupied lattice sites with two-component bosons can be described by an anisotropic spin-1 Heisenberg model, in which the exchange term is isotropic and an uniaxial anistropy term proportional to $(S_z)^2$ is added. In this Letter, we initialize and evolve in time a spin-rotated state prepared from spin-polarized Mott-insulating and observe that the average pairing between two different components evolves according to dynamics predicted by the anisotropic spin-1 Heisenberg model. In particular, we predict and observe that the change in the average pairing shows a resonant behavior. This behavior can be qualitatively described by a simple two-level picture. 
\end{abstract}

\maketitle

Outline:
\begin{itemize}
    \item Introduction: spin models, anisotropic spin-1 Heisenberg Hamiltonian
    \item Idealized two-site model, nonequilibrium dynamics, thermalization
    \item Generalizing to many sites (hammer home the message that the initial behavior is described by $J$ between nearest-neighbors).
    \item Experimental setup \& results
\end{itemize}

Potential sections for supplemental material:
\begin{itemize}
    \item Further details on the two-site toy model
    \item Background on doublon statistics reconstruction (Feshbach detection)
    \item Mention noise removal algorithm?
    \item Details of MPS calculation
    \item Derivation of anisotropic spin-1 effective Hamiltonian
\end{itemize}

Nomenclature
\begin{itemize}
    \item ``nonlinear-u term'': it seems ``uniaxial single-ion anisotropy'' is a oft-used term in condensed matter physics \cite{WANG20091904, He2007}.
\end{itemize}

Some papers that might be useful while we're figuring out how we're building on previous work:
\begin{itemize}
    \item `Out-of-equilibrium quantum magnetism and thermalization in a spin-3 many-body dipolar lattice system' \cite{Lepoutre2019}. Here they look at spin dynamics implemented using the dipolar interaction between Cr atoms. It maps onto an \emph{XXZ} Heisenberg model. They discuss entanglement generation, which may be of interest if we choose to mention that.
    
    \item `Nonequilibrium Quantum Magnetism in a Dipolar Lattice Gas' \cite{dePaz2013}. An older paper by the same group where they use a lattice gas to explore out-of-equilibrium dynamics in the \emph{t-J} model using the same platform.
    
    \item In case we're looking for a wider context to our Hamiltonian: `Magnetism and the effect of anisotropy with a one-dimensional monatomic chain of cobalt using a Monte Carlo simulation' \cite{He2007}. They use a Hamiltonian that's very similar to ours to understand the physics of ferromagnetic one-dimensional chains of Co atoms deposited on a substrate.
\end{itemize}
\bibliography{spdf_oscillation.bib}
\end{document}

